\documentclass[a4paper,14pt]{article} % формат документа

\usepackage{cmap} % поиск в ПДФ
\usepackage[T2A]{fontenc} % кодировка
\usepackage[utf8]{inputenc} % кодировка исходного текста
\usepackage[english,russian]{babel} % локализация и переносы
\usepackage[left = 2cm, right = 1cm, top = 2cm, bottom = 2 cm]{geometry} % поля
\usepackage{listings}
\usepackage{graphicx} % для вставки рисунков
\graphicspath{{pictures/}}
\DeclareGraphicsExtensions{.pdf,.png,.jpg}
\newcommand{\anonsection}[1]{\section*{#1}\addcontentsline{toc}{section}{#1}}

\lstset{ 
	language=C++,  % Язык программирования 
	numbers=left,  % С какой стороны нумеровать          
	frame=single,  % Добавить рамку
}

\begin{document}
	\begin{titlepage}
	
	\end{titlepage}
    
	\tableofcontents
	
	\newpage
    
	\anonsection{Введение}
	
	\hfill
	
	В данной работе описывается моделирование взрыва большого числа частиц, при столкновении с телом, имеющим больший размер. Моделирование основано на физическом явлении взрыва взрыва и возникающих побочных эффектов. 
	
	Взрывы являются одними из самых визуально захватывающих явлений, известных человечеству. Они широко используются в компьютерных играх и фильмах-боевиках. Среды, разработанные для учебных симуляций, часто фокусируются на насильственных или иных опасных ситуациях, поэтому они также включают в себя некоторые формы взрывных явлений. Компьютерное моделирование этого мощного события гораздо дешевле, быстрее и безопаснее, чем проведение реального эксперимента. 
	
	Объекты в сцене представлены в виде твердых частиц, которые приводятся в движение различными силами, такими как сила тяжести, сила реакции опоры, гравитационная сила взаимодействия частиц. 
		
	\newpage

	\section{Аналитическая часть}
	
	\hfill
	
	Системы частиц - это метод, используемый для моделирования объектов, которые трудно моделировать с помощью классического поверхностного моделирования. Объекты, которые являются динамическими, изменяются во времени или являются нечеткими, текучими или другими нестатическими способами, могут быть смоделированы более реалистичным способом с использованием систем частиц. Системы частиц создают объем частиц с индивидуальными свойствами, а не поверхность с текстурой. При таком подходе возможно изменять внешний вид каждой отдельной частицы с течением времени и, таким образом, создавать визуализации, которые реалистично отображают огонь, дым, воду, взрывы и с некоторыми изменениями даже волосы и мех. Они также используются для представления неестественных явлений, таких как магические эффекты. 
Эта техника имеет много полезных приложений, и одно из них в спецэффектах. Будь то большие взрывы, пыль, пожары или большие толпы людей, спецэффекты необходимы во многих фильмах, анимациях, съемках или компьютерных играх. Фильмы, показывающие массивные взрывы и разрушенные здания, могут захотеть имитировать эти эффекты, а не делать их по-настоящему. В этом отчете мы сосредоточимся на моделировании .... Взрыв в реальности состоит из мелких частиц, которые движутся независимо друг от друга, однако, поскольку частицы будут взаимодействовать в основном одинаковыми силами, они будут вести себя одинаково.

	Системы частиц были используются для моделирования различных явлений: от огня и дыма до структур волос, ткани и меха. В сущности, системы частиц лучше всего применять к явлениям, которые можно рассматривать как объем крошечных объектов, которые в основном взаимодействуют со своими соседями. Говорят, что системы частиц использовались в видеоиграх еще в 1960-х годах для имитации взрывов с использованием 2D-пиксельных облаков. Однако в начале 1980-х годов Уильям Т. Ривз использовал системы частиц для моделирования огня, поглощающего планету, в фильме «Звездный путь II: Гнев Хана». Ривз считается изобретателем систем частиц, и в 1983 году он опубликовал статью, объясняющую, как это было сделано. Ривз продолжил развивать эту идею и с тех пор вносил больший вклад в область систем частиц. Разработка продолжалась, и сегодня большинство игровых движков содержат систему частиц. 
	
	Визуализация взрыва частиц может быть описана по разному в зависимости от источника силы, порождающий взрыв. 
В данной работе, источником является шарообразное тело, врезающееся в систему более мелких частиц.
	
	Моделирование в виртуальных пикселях уже почти не встречается в производстве 3D графики. Сейчас при 3D моделировании объекты часто создают в основном только двумя способами:
	\begin{enumerate}
	\item либо с помощью плоских полигонов — тем самым будет создана полая модель без внутреннего наполнения, но для тех, кто наблюдает 3D, часто и не требуется знать, что, например, у 3D кошки внутри ничего нет. 
	\item Для наблюдателя достаточно лишь хорошо сшитой из треугольных полигонов поверхности кошки.
либо с помощью объёмных кубиков — вокселей, которые полностью заполняют внутренности 3D модели, где каждый такой кубик несёт в себе информацию о том, чем он является, например, кожей, мышцами, костями и т.д.
	\end{enumerate}

	Вексель — образовано из слов: объёмный и пиксель — элемент объёмного изображения, содержащий значение элемента растра в трёхмерном пространстве. Вокселы являются аналогами двумерных пикселей для трёхмерного пространства. Воксельные модели часто используются для визуализации и анализа медицинской и научной информации.
	
	В компьютерной графике воксели используются как альтернатива полигонам. Виртуальными пикселями «рисуют» плоские 2D модели объектов в 3D мире.

	То есть в отличие от полигонов и пикселей воксели — это истинный 3D кирпичик, а не 2D плоскость, которая «окружает» пустое 3D пространство.

	Ввиду того, что полигональные модели пусты по своей природе, очень трудно моделировать их поведение в 3D мире. Например, если программисту нужно смоделировать поведение воды в 3D игре про пиратов, он сталкивается с проблемой: как смоделировать волны на поверхности воды? Как моделировать всплески воды, ведь вода в игре — это просто ковёр, сплетённый из треугольников голубого цвета, под этой плоскостью ничего нет, а между тем нужно показать пенящуюся и плескающуюся воду? То есть надо показать отделение частей воды друг от друга в виде пены и всплесков. Приходится вводить новые объекты в память компьютера, а управление этими дополнительными объектами требует большого искусства именно от программиста, а не от дизайнера.

	Если же воду моделировать через воксели, то всё становится гораздо проще, ибо вся вода от поверхности океана и до дна состоит из «атомов», которые легко «отделяются» друг от друга естественным, с точки зрения программиста, путём.

	На сцене располагается неподвижная, в начальный момент, группа частиц, расположенных вместе, шарообразное подвижное тело, точечный источник света.
В начальный момент времени, есть возможность скорректировать размеры тела, траекторию его движения, поменять расположение камеры.
После нажатия на кнопку, произойдёт симуляция.
        
	\newpage

	\section{Конструкторская часть}
	
	\hfill
	
	В начале необходимо создать воксельный движок, удовлетворяющий следующим требованиям:
	\begin{enumerate}
	\item[-] Эффективный - способен отображать большое количество вокселей на экране одновременно.

	\item[-] Динамический - движок вокселей должен иметь возможность изменять любой воксель в любое время.

	\item[-] Экспансивный - масштаб визуализации должен быть большим и не ограничиваться произвольными ограничениями.
	\end{enumerate}
	
	\newpage
	\section{Технологическая часть}
	\hfill
	\newpage
        
	\section{Экспериментальная часть}
	\hfill
	\newpage

	\anonsection{Заключение}
	\hfill
 	\newpage

	\begin{thebibliography}{}
        		\bibitem{}
	\end{thebibliography} 

\end{document}
