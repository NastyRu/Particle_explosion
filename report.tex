\documentclass[a4paper,14pt]{article} % формат документа

\usepackage{cmap} % поиск в ПДФ
\usepackage[T2A]{fontenc} % кодировка
\usepackage[utf8]{inputenc} % кодировка исходного текста
\usepackage[english,russian]{babel} % локализация и переносы
\usepackage[left = 2cm, right = 1cm, top = 2cm, bottom = 2 cm]{geometry} % поля
\usepackage{listings}
\usepackage{graphicx} % для вставки рисунков
\graphicspath{{pictures/}}
\DeclareGraphicsExtensions{.pdf,.png,.jpg}
\newcommand{\anonsection}[1]{\section*{#1}\addcontentsline{toc}{section}{#1}}

\lstset{ 
	language=C++,  % Язык программирования 
	numbers=left,  % С какой стороны нумеровать          
	frame=single,  % Добавить рамку
}

\begin{document}
	\begin{titlepage}
	
	\end{titlepage}
    
	\tableofcontents
	
	\newpage
    
	\anonsection{Введение}
	
	\hfill
	
	На протяжении десятилетий взрывы были самыми динамичными и визуально привлекательными спецэффектами в кино и видеоиграх. Они стали настолько заметными в боевиках и приключенческих фильмах, что кажется необычным, когда его нет в фильме. Каким был бы фильм "Звездные войны" без финального взрыва Звезды Смерти? 
	
	Традиционно взрывные эффекты создаются перед камерой, а не в компьютере. Либо строится уменьшенная модель и взрывается перед высокоскоростными камерами, либо используются настоящие взрывчатые вещества. 
	
	Правда в кинематографе, важна эффектность, поэтому все заранее планируется в мельчайших подробностях.
	
	Если нужно взорвать настоящее здание, пусть даже специально для этого построенное, его предварительно готовят: подпиливают рамы так, чтобы нужная часть их осталась на месте, а нужная – разлетелась на части. Также это делается для того, чтобы здание развалилось при минимально необходимом заряде. Все части, что должны развалиться, делаются из легких материалов с тем, чтобы если какие-то куски случайно упадут на голову кинематографистам, ущерб был бы минимален. 
	
	Если при этом в кадре должны быть персонажи, с людьми скрупулезно отрабатываются все их передвижения, фиксируется время, необходимое для того, чтобы они достигли безопасного расстояния до того, как произойдет главный взрыв. [1]
	
	Существует множество веских причин для использования компьютеров для создания взрывных эффектов вместо более традиционных практических методов. Основной мотивацией, конечно, забота о безопасности актеров. Когда взрыв происходит полностью внутри компьютера, нет никаких шансов на то, что кто-то случайно попадет в зону взрыва. Также компьютерные взрывы дешевле и быстрее, чем точное масштабирование и размещение детонаторов, а также огнестойкость существующих конструкций или создание специальных миниатюр. Когда режиссеры снимают практический взрыв, они настраиваются на несколько дней, чтобы получить несколько ракурсов на один единственный взрыв. С помощью компьютерных эффектов режиссеры могут посмотреть на промежуточный результат и попросить что-то изменить, чтобы более точно отразить свое творческое видение. 
	
	Объектом исследования является создание максимально приближенной модели взрыва большого числа частиц, при столкновении с телом, имеющим больший размер с использованием графического редактора систем частиц. Моделирование основано на физическом явлении взрыва взрыва и возникающих побочных эффектов. 
	
	Цель исследования состоит в разработке системы компонентов моделирующих взаимодействие частиц в заданном пространстве за заданное время и взаимодействующих с окружающей средой. 
	
	Для достижения цели решаются следующие исследовательские задачи.
	\begin{enumerate}
	\item Определить понятие системы частиц. 
	\item Создать движок для работы с частицами. 
	\item Изучить физическое явление - взрыв. 
	\item Смоделировать взрыва большого числа частиц, при столкновении с телом
	\end{enumerate}
		
	\newpage

	\section{Аналитическая часть}
	
	\hfill
	
	Целью работы является создание максимально приближенной модели взрыва большого числа частиц. Объекты в сцене представлены в виде твердых частиц, которые приводятся в движение различными силами, такими как сила тяжести, сила реакции опоры, гравитационная сила взаимодействия частиц. 
	
	\subsection{Описание задачи}
	\hfill
	
	Системы частиц - это метод, используемый для моделирования объектов, которые трудно моделировать с помощью классического поверхностного моделирования. Объекты, которые являются динамическими, изменяются во времени или являются нечеткими, текучими или другими нестатическими способами, могут быть смоделированы более реалистичным способом с использованием систем частиц. Системы частиц создают объем частиц с индивидуальными свойствами, а не поверхность с текстурой. При таком подходе возможно изменять внешний вид каждой отдельной частицы с течением времени и, таким образом, создавать визуализации, которые реалистично отображают огонь, дым, воду, взрывы и с некоторыми изменениями даже волосы и мех. Они также используются для представления неестественных явлений, таких как магические эффекты. 
	
	Уже в начале 1980-х годов Уильям Т. Ривз использовал системы частиц для моделирования огня, поглощающего планету, в фильме «Звездный путь II: Гнев Хана». Ривз считается изобретателем систем частиц, и в 1983 году он опубликовал статью, объясняющую, как это было сделано. Ривз продолжил развивать эту идею и с тех пор вносил больший вклад в область систем частиц. Разработка продолжалась, и сегодня большинство игровых движков содержат систему частиц. [2]
	
	Эта техника имеет много полезных приложений, будь то большие взрывы, пыль, пожары или большие толпы людей, спецэффекты необходимы во многих фильмах, анимациях, съемках или компьютерных играх. Фильмы, показывающие массивные взрывы и разрушенные здания, могут имитировать эти эффекты, а не создавать их по-настоящему. В этой работе мы сосредоточимся на моделировании взрыва большого числа частиц, при столкновении с телом, имеющим больший размер.
	
	Взрыв в реальности состоит из мелких частиц, которые движутся независимо друг от друга, однако, поскольку частицы будут взаимодействовать в основном одинаковыми силами, они будут вести себя одинаково.
	
	Визуализация взрыва частиц может быть описана по разному в зависимости от источника силы, порождающий взрыв. В данной работе, источником является шарообразное тело, врезающееся в систему более мелких частиц.
	
	На сцене располагается неподвижная, в начальный момент, группа частиц, расположенных вместе, шарообразное подвижное тело, точечный источник света.
	
	В начальный момент времени, есть возможность скорректировать размеры тела, траекторию его движения, поменять расположение камеры.
	
	После нажатия на кнопку, произойдёт симуляция.
	
	\subsection{Пути решения}
	\hfill
	
	В настоящее время при 3D моделировании объекты часто создают в основном только двумя способами. 
	\begin{enumerate}
	\item Либо с помощью плоских полигонов — тем самым будет создана полая модель без внутреннего наполнения. 
	\item Либо с помощью объёмных кубиков — вокселей, которые полностью заполняют внутренности 3D модели, где каждый такой кубик несёт в себе информацию о том, чем он является.
	\end{enumerate}

	Вексель -- образовано из слов: объёмный и пиксель — элемент объёмного изображения, содержащий значение элемента растра в трёхмерном пространстве. Воксели являются аналогами двумерных пикселей для трёхмерного пространства. 

	Ввиду того, что полигональные модели пусты по своей природе, очень трудно моделировать их поведение в 3D мире, как, например, всплески воды. Если же воду моделировать через воксели, то всё становится гораздо проще, так как вся вода от поверхности океана и до дна состоит из атомов, которые можно рассматривать, как набор отдельных частиц.
	
	В начале необходимо создать воксельный движок, удовлетворяющий следующим требованиям:
	\begin{enumerate}
	\item[-] Эффективный - способен отображать большое количество вокселей на экране одновременно.

	\item[-] Динамический - движок вокселей должен иметь возможность изменять любой воксель в любое время.

	\item[-] Экспансивный - масштаб визуализации должен быть большим и не ограничиваться произвольными ограничениями.
	\end{enumerate}

	\subsection{Выводы}
	\hfill
        
        
	\newpage

	\section{Конструкторская часть}
	
	\hfill
	
	
	
	\newpage
	\section{Технологическая часть}
	\hfill
	\newpage
        
	\section{Экспериментальная часть}
	\hfill
	\newpage

	\anonsection{Заключение}
	\hfill
 	\newpage

	\begin{thebibliography}{}
        		\bibitem{} Взрывы в кино: что за кадром? [Электронный ресурс]. -- Режим доступа: http://www.mir3d.ru/vfx/950/ (дата обращения: 10.10.2019).
		\bibitem{} William T. Reeves. Particle Systems – A Technique for Modeling a Class of Fuzzy Objects. [Текст]  – Computer Graphics, том 17, №3, с. 359-376, 1983 г.
	\end{thebibliography} 

\end{document}
