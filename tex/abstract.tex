% Также можно использовать \Referat, как в оригинале
\Referat

\hfill

Курсовой проект представляет собой приложение для визуализации взаимодействия большого числа частиц при столкновении, взрыв. 

В данной работе рассматриваются алгоритм трассировки лучей, тени и диффузное отражение, расчитываются параметры физической модели. Затем данные алгоритмы реализуются, рассматриваются параметры и примеры работы приложения, реализованного на языке программирования С++ с использованием кроссплатформенной библиотеки Qt. 

Полученная в результате работы визуализация может быть использована в различных фильмах или играх для замены реальных взрывов. 

Отчёт содержит \pageref{LastPage}\,~страницу%
    \ifnum \totfig >0
    , \totfig~рисунок%
    \fi
    \ifnum \tottab >0
    , \tottab~таблицу%
    \fi
    %
    \ifnum \totbib >0
    , \totbib~источников%
    \fi
    %
    \ifnum \totapp >0
    , \totapp~прил.%
    \else
    .%
    \fi

%%% Local Variables: 
%%% mode: latex
%%% TeX-master: "rpz"
%%% End: 
