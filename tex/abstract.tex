% Также можно использовать \Referat, как в оригинале
\Referat

\hfill

В данной работе проводится исследование взаимодействия большого числа частиц при столкновении и разработка приложения для визуализации взаимодействия, взрыва частиц. 

Целью данной работы является изучение физической модели взрыва и ее реализации с помощью методов компьютерной графики: системы частиц, трассировки лучей. В результате получено приложение написанное на С++ с использованием кроссплатформенной библиотеки Qt. 

В данной работе рассматривается алгоритм трассировки лучей, тени и диффузное отражение, рассчитываются параметры физической модели. Затем данные алгоритмы реализуются, рассматриваются параметры и примеры работы приложения. 

Полученная в результате работы визуализация, может быть использована в различных фильмах или играх, для замены реальных взрывов. 

Отчет содержит \pageref{LastPage}\,~стр.%
    \ifnum \totfig >0
    , \totfig~рис.%
    \fi
    \ifnum \tottab >0
    , \tottab~табл.%
    \fi
    %
    \ifnum \totbib >0
    , \totbib~источн.%
    \fi
    %
    \ifnum \totapp >0
    , \totapp~прил.%
    \else
    .%
    \fi

%%% Local Variables: 
%%% mode: latex
%%% TeX-master: "rpz"
%%% End: 
