\chapter{Аналитический раздел}
\hfill

Целью работы является создание максимально приближенной модели взрыва большого числа частиц. Объекты в сцене представлены в виде твердых частиц, которые приводятся в движение различными силами.  

\section{Анализ предметной области }

Система частиц – широко используемый в компьютерной графике метод представления объектов, не имеющих четки геометрических границ. Облака, туманности, дым, взрыв, снег – все эти объекты моделируются с помощью систем частиц. \cite{definition}


\section{Обзор и анализ существующих решений, обоснование необходимости разработки}

Актуальность темы исследования обусловлена тем, что за последние несколько лет технология виртуальной реальности совершила огромный скачок в развитии и расширении сфер применения. Если раньше эта технология в основном применялась в военной промышленности и компьютерных играх, то сейчас виртуальная реальность проникает практически во все сферы деятельности человека: медицину, образование, архитектуру, рекламу и прочее. Эта технология имеет огромный потенциал и поэтому она так активно развивается.

\section{Выбор, обоснование метода моделирования и алгоритма}
