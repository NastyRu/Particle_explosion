%% Преамбула TeX-файла

% 1. Стиль и язык
\documentclass[utf8x, 14pt]{G7-32} % Стиль (по умолчанию будет 14pt)

% Остальные стандартные настройки убраны в preamble.inc.tex.
\sloppy

% Настройки стиля ГОСТ 7-32
% Для начала определяем, хотим мы или нет, чтобы рисунки и таблицы нумеровались в пределах раздела, или нам нужна сквозная нумерация.
\EqInChapter % формулы будут нумероваться в пределах раздела
\TableInChapter % таблицы будут нумероваться в пределах раздела
\PicInChapter % рисунки будут нумероваться в пределах раздела

% Добавляем гипертекстовое оглавление в PDF
\usepackage[
bookmarks=true, colorlinks=true, unicode=true,
urlcolor=black,linkcolor=black, anchorcolor=black,
citecolor=black, menucolor=black, filecolor=black,
]{hyperref}

\AfterHyperrefFix

\usepackage{microtype}% полезный пакет для микротипографии, увы под xelatex мало чего умеет, но под pdflatex хорошо улучшает читаемость

% Тире могут быть невидимы в Adobe Reader
\ifInvisibleDashes
\MakeDashesBold
\fi

% С такими оно полями оно работает по-умолчанию:
% \RequirePackage[left=20mm,right=10mm,top=20mm,bottom=20mm,headsep=0pt,includefoot]{geometry}
% Если вас тошнит от поля в 10мм --- увеличивайте до 20-ти, ну и про переплёт не забывайте:
\geometry{right=20mm}
\geometry{left=30mm}
\geometry{bottom=20mm}
\geometry{ignorefoot}% считать от нижней границы текста


% Пакет Tikz
\usepackage{tikz}
\usetikzlibrary{arrows,positioning,shadows}

% Произвольная нумерация списков.
\usepackage{enumerate}

% ячейки в несколько строчек
\usepackage{multirow}

% itemize внутри tabular
\usepackage{paralist,array}

%\setlength{\parskip}{1ex plus0.5ex minus0.5ex} % разрыв между абзацами
\setlength{\parskip}{1ex} % разрыв между абзацами
\usepackage{blindtext}

% Центрирование подписей к плавающим окружениям
%\usepackage[justification=centering]{caption}

\usepackage{newfloat}
\DeclareFloatingEnvironment[
placement={!ht},
name=Equation
]{eqndescNoIndent}
\edef\fixEqndesc{\noexpand\setlength{\noexpand\parindent}{\the\parindent}\noexpand\setlength{\noexpand\parskip}{\the\parskip}}
\newenvironment{eqndesc}[1][!ht]{%
    \begin{eqndescNoIndent}[#1]%
\fixEqndesc%
}
{\end{eqndescNoIndent}}

\usepackage{float}

\usepackage{graphicx} % для вставки рисунков
\graphicspath{{image/}}
\DeclareGraphicsExtensions{.pdf, .png, .jpg}


% Настройки листингов.
\ifPDFTeX
% 8 Листинги

\usepackage{listings}

% Значения по умолчанию
\lstset{
  basicstyle= \footnotesize,
  breakatwhitespace=true,% разрыв строк только на whitespacce
  breaklines=true,       % переносить длинные строки
%   captionpos=b,          % подписи снизу -- вроде не надо
  inputencoding=koi8-r,
  numbers=left,          % нумерация слева
  numberstyle=\footnotesize,
  showspaces=false,      % показывать пробелы подчеркиваниями -- идиотизм 70-х годов
  showstringspaces=false,
  showtabs=false,        % и табы тоже
  stepnumber=1,
  tabsize=4,              % кому нужны табы по 8 символов?
  frame=single
}

% Стиль для псевдокода: строчки обычно короткие, поэтому размер шрифта побольше
\lstdefinestyle{pseudocode}{
  basicstyle=\small,
  keywordstyle=\color{black}\bfseries\underbar,
  language=Pseudocode,
  numberstyle=\footnotesize,
  commentstyle=\footnotesize\it
}

% Стиль для обычного кода: маленький шрифт
\lstdefinestyle{realcode}{
  basicstyle=\scriptsize,
  numberstyle=\footnotesize
}

% Стиль для коротких кусков обычного кода: средний шрифт
\lstdefinestyle{simplecode}{
  basicstyle=\footnotesize,
  numberstyle=\footnotesize
}

% Стиль для BNF
\lstdefinestyle{grammar}{
  basicstyle=\footnotesize,
  numberstyle=\footnotesize,
  stringstyle=\bfseries\ttfamily,
  language=BNF
}

% Определим свой язык для написания псевдокодов на основе Python
\lstdefinelanguage[]{Pseudocode}[]{Python}{
  morekeywords={each,empty,wait,do},% ключевые слова добавлять сюда
  morecomment=[s]{\{}{\}},% комменты {а-ля Pascal} смотрятся нагляднее
  literate=% а сюда добавлять операторы, которые хотите отображать как мат. символы
    {->}{\ensuremath{$\rightarrow$}~}2%
    {<-}{\ensuremath{$\leftarrow$}~}2%
    {:=}{\ensuremath{$\leftarrow$}~}2%
    {<--}{\ensuremath{$\Longleftarrow$}~}2%
}[keywords,comments]

% Свой язык для задания грамматик в BNF
\lstdefinelanguage[]{BNF}[]{}{
  morekeywords={},
  morecomment=[s]{@}{@},
  morestring=[b]",%
  literate=%
    {->}{\ensuremath{$\rightarrow$}~}2%
    {*}{\ensuremath{$^*$}~}2%
    {+}{\ensuremath{$^+$}~}2%
    {|}{\ensuremath{$|$}~}2%
}[keywords,comments,strings]

% Подписи к листингам на русском языке.
\renewcommand\lstlistingname{Листинг}
\renewcommand\lstlistlistingname{Листинги}

\else
\usepackage{local-minted}
\fi

% Полезные макросы листингов.
% Любимые команды
\newcommand{\Code}[1]{\textbf{#1}}


% Стиль титульного листа и заголовки

%\NirEkz{Экз. 3}                                  % Раскоментировать если не требуется
%\NirGrif{Секретно}                % Наименование грифа

%\gosttitle{Gost7-32}       % Шаблон титульной страницы, по умолчанию будет ГОСТ 7.32-2001, 
% Варианты GostRV15-110 или Gost7-32 
 
\NirOrgLongName{Федеральное государственное бюджетное образовательное учреждение
 \\ высшего профессионального образования\par
«Московский государственный технический университет имени Н.Э. Баумана» \\ (МГТУ им. Н.Э. Баумана)
}                                           %% Полное название организации

\NirUdk{УДК № 378.14}
\NirGosNo{№ госрегистрации 01970006723}
%\NirInventarNo{Инв. № ??????}

%\NirConfirm{Согласовано}                  % Смена УТВЕРЖДАЮ
%\NirBoss[.49]{Проректор университета\\по научной работе}{Н.С. Жернаков}            %% Заказчик, утверждающий НИР


%\NirReportName{Научно-технический отчет}   % Можно поменять тип отчета
%\NirAbout{О составной части \par опытно-конструкторской работы} %Можно изменить о чем отчет

%\NirPartNum{Часть}{1}                      % Часть номер

%\NirBareSubject{}                  % Убирает по теме если раскоментить

% \NirIsAnnotacion{АННОТАЦИОННЫЙ }         %% Раскомментируйте, если это аннотационный отчёт
%\NirStage{промежуточный}{Этап \No 1}{} %%% Этап НИР: {номер этапа}{вид отчёта - промежуточный или заключительный}{название этапа}
%\NirStage{}{}{} %%% Этап НИР: {номер этапа}{вид отчёта - промежуточный или 

%\Nir{Социально-экономические проблемы подготовки военных специалистов\\в гражданских вузах России}

\NirSubject{ Моделирование взрыва частиц }                                   % Наименование темы
%\NirFinal{}                        % Заключительный, если закоментировать то промежуточный
%\finalname{итоговый}               % Название финального отчета (Заключительный) 
%\NirCode{Шифр\,---\,САПР-РЛС-ФИЗТЕХ-1} % Можно задать шифр как в ГОСТ 15.110
\NirCode{}

%\NirManager{Зам. проректора по научной работе}{Р.А. Бадамшин  } %% Название руководителя
\NirIsp{Руководитель проекта}{А.С. Кострицкий} %% Название руководителя

%\NirYear{1999}%% если нужно поменять год отчёта; если закомментировано, ставится текущий год
\NirTown{Москва}                           %% город, в котором написан отчёт



\begin{document}

\frontmatter % выключает нумерацию ВСЕГО; здесь начинаются ненумерованные главы: реферат, введение, глоссарий, сокращения и прочее.

\maketitle %создает титульную страницу


%\listoffigures                         % Список рисунков

%\listoftables                          % Список таблиц

%\NormRefs % Нормативные ссылки 
% Команды \breakingbeforechapters и \nonbreakingbeforechapters
% управляют разрывом страницы перед главами.
% По-умолчанию страница разрывается.

% \nobreakingbeforechapters
% \breakingbeforechapters

\tableofcontents

% Также можно использовать \Referat, как в оригинале
\begin{abstract}

    Отчет содержит \pageref{LastPage}\,стр.%
    \ifnum \totfig >0
    , \totfig~рис.%
    \fi
    \ifnum \tottab >0
    , \tottab~табл.%
    \fi
    %
    \ifnum \totbib >0
    , \totbib~источн.%
    \fi
    %
    \ifnum \totapp >0
    , \totapp~прил.%
    \else
    .%
    \fi
\end{abstract}

%%% Local Variables: 
%%% mode: latex
%%% TeX-master: "rpz"
%%% End: 


\printnomenclature % Автоматический список сокращений

\Introduction

\hfill

	На протяжении десятилетий взрывы были самыми динамичными и визуально привлекательными спецэффектами в кино и видеоиграх. Они стали настолько заметными в боевиках и приключенческих фильмах, что кажется необычным, когда его нет в фильме. Каким был бы фильм "Звездные войны"  без финального взрыва Звезды Смерти? 
	
	Традиционно взрывные эффекты создаются перед камерой, а не в компьютере. Либо строится уменьшенная модель и взрывается перед высокоскоростными камерами, либо используются настоящие взрывчатые вещества. 
	
	Правда в кинематографе, важна эффектность, поэтому все заранее планируется в мельчайших подробностях.
	
	Если нужно взорвать настоящее здание, пусть даже специально для этого построенное, его предварительно готовят: подпиливают рамы так, чтобы нужная часть их осталась на месте, а нужная – разлетелась на части. Также это делается для того, чтобы здание развалилось при минимально необходимом заряде. Все части, что должны развалиться, делаются из легких материалов с тем, чтобы если какие-то куски случайно упадут на голову кинематографистам, ущерб был бы минимален. 
	
	Если при этом в кадре должны быть персонажи, с людьми скрупулезно отрабатываются все их передвижения, фиксируется время, необходимое для того, чтобы они достигли безопасного расстояния до того, как произойдет главный взрыв. \cite{cinemaexplosion}
	
	Существует множество веских причин для использования компьютеров для создания взрывных эффектов вместо более традиционных практических методов. Основной мотивацией, конечно, забота о безопасности актеров. Когда взрыв происходит полностью внутри компьютера, нет никаких шансов на то, что кто-то случайно попадет в зону взрыва. Также компьютерные взрывы дешевле и быстрее, чем точное масштабирование и размещение детонаторов, а также огнестойкость существующих конструкций или создание специальных миниатюр. Когда режиссеры снимают практический взрыв, они настраиваются на несколько дней, чтобы получить несколько ракурсов на один единственный взрыв. С помощью компьютерных эффектов режиссеры могут посмотреть на промежуточный результат и попросить что-то изменить, чтобы более точно отразить свое творческое видение. 
	
	Целью проекта является создание максимально приближенной модели взрыва большого числа частиц, при столкновении с телом, имеющим больший размер с использованием графического редактора систем частиц. Моделирование основано на физическом явлении взрыва взрыва и возникающих побочных эффектов в заданном пространстве за заданное время и взаимодействующих с окружающей средой. 
	
	Для достижения поставленной цели необходимо решить следующие задачи:
	\begin{enumerate}
		\item Определить понятие системы частиц. 
		\item Создать движок для работы с частицами. 
		\item Изучить физическое явление - взрыв. 
		\item Смоделировать взрыва большого числа частиц, при столкновении с телом
	\end{enumerate}





\mainmatter % это включает нумерацию глав и секций в документе ниже

\chapter{Аналитический раздел}
\hfill

Целью работы является создание максимально приближенной модели взрыва большого числа частиц. Объекты в сцене представлены в виде твердых частиц, которые приводятся в движение различными силами.  

\section{Анализ предметной области }

Система частиц – широко используемый в компьютерной графике метод представления объектов, не имеющих четки геометрических границ. Облака, туманности, дым, взрыв, снег – все эти объекты моделируются с помощью систем частиц. \cite{definition}


\section{Обзор и анализ существующих решений, обоснование необходимости разработки}

Актуальность темы исследования обусловлена тем, что за последние несколько лет технология виртуальной реальности совершила огромный скачок в развитии и расширении сфер применения. Если раньше эта технология в основном применялась в военной промышленности и компьютерных играх, то сейчас виртуальная реальность проникает практически во все сферы деятельности человека: медицину, образование, архитектуру, рекламу и прочее. Эта технология имеет огромный потенциал и поэтому она так активно развивается.

\section{Выбор, обоснование метода моделирования и алгоритма}

\chapter{Конструкторский раздел}

\chapter{Технологический раздел}

\chapter{Экспериментальный раздел}

\include{60-economics}
\include{70-bzd}

\backmatter %% Здесь заканчивается нумерованная часть документа и начинаются ссылки и
            
\Conclusion % заключение к отчёту

В результате проделанной работы 

%%% Local Variables: 
%%% mode: latex
%%% TeX-master: "rpz"
%%% End: 
%% заключение


% % Список литературы при помощи BibTeX
% Юзать так:
%
% pdflatex rpz
% bibtex rpz
% pdflatex rpz

\bibliographystyle{ugost2008}
\bibliography{rpz}
\nocite{*}
%%% Local Variables: 
%%% mode: latex
%%% TeX-master: "rpz"
%%% End: 



\appendix   % Тут идут приложения

\chapter{Картинки}


%%% Local Variables: 
%%% mode: latex
%%% TeX-master: "rpz"
%%% End: 


\end{document}

%%% Local Variables:
%%% mode: latex
%%% TeX-master: t
%%% End:
