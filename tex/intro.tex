\Introduction

\hfill

	Взрывы --- динамичные и визуально привлекательные спецэффекты в кино и видеоиграх. Они стали настолько заметными в боевиках и приключенческих фильмах, что кажется необычным, когда его нет в фильме. Каким был бы фильм <<Звездные войны>> без финального взрыва Звезды Смерти? 
	
	Традиционно взрывы создаются перед камерой, а не в компьютере. Либо строится уменьшенная модель и взрывается перед высокоскоростными камерами, либо используются настоящие взрывчатые вещества. 
	
	Однако, в кинематографе важна эффектность, поэтому все заранее планируется в мельчайших подробностях.
	
	Если нужно взорвать настоящее здание, пусть даже специально для этого построенное, его предварительно готовят: подпиливают рамы так, чтобы нужная часть их осталась на месте, а нужная --- разлетелась на части при минимально необходимом заряде. Все части, что должны развалиться, делаются из легких материалов, чтобы если какие-то куски случайно упадут на голову кинематографистам, ущерб был бы минимален. 
	
	Если при этом в кадре должны быть персонажи, с людьми отрабатываются все их передвижения, фиксируется время, необходимое для того, чтобы они достигли безопасного расстояния до того, как произойдет главный взрыв. \cite{cinemaexplosion}
	
	Существует множество веских причин использования компьютеров для создания взрывных эффектов вместо более традиционных практических методов. Основная мотивация --- забота о безопасности актеров. Когда взрыв происходит полностью внутри компьютера, нет никаких шансов на то, что кто-то случайно попадет в зону взрыва. Когда режиссеры снимают практический взрыв, они настраиваются несколько дней, чтобы получить несколько ракурсов на один единственный взрыв. С помощью компьютерных эффектов режиссеры могут посмотреть на промежуточный результат и попросить что-то изменить, чтобы более точно отразить свое творческое видение. 
	
	Целью проекта является создание максимально приближенной модели взрыва большого числа частиц, при столкновении с телом. Моделирование основано на физическом явлении взрыва и возникающих побочных эффектов в заданном пространстве за заданное время и взаимодействующих с окружающей средой. 
	
	Для достижения поставленной цели необходимо решить следующие задачи:
	\begin{enumerate}
		\item Определить понятие системы частиц. 
		\item Создать движок для работы с частицами. 
		\item Изучить физическое явление - взрыв. 
		\item Смоделировать взрыв большого числа частиц, при столкновении с телом. 
	\end{enumerate}



